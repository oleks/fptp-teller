\section*{Formal methods}

% new concepts, or concepts that the reader may understand differently are
% emphasized.

A \emph{system} is a set of data structures and transformations for solving a
particular set of problems. A system \emph{specification} is a document
describing the system. A \emph{specification language} is the language used to
write a specification.

Every program is a specification, and every programming language is a
specification language. A specification is \emph{efficient} if it is directly
executable by a computer. A specification need not be efficient in general.

For instance, the specification language may allow for the use of certain
mathematical constructs. This would allow to disregard certain data
representation and transformation details. These details are important for an
efficient implementation, but may get in the way of reasoning about system
\emph{correctness}.

We reason about system correctness, by stating certain \emph{properties} that
should hold, and \emph{verifying} that they indeed do.

One way to do this is to verify system behaviour. That is, for all possible
input and generate the expected output, and check that the system produces the
same result. As the problem space explodes (as it quickly does), it becomes
infeasible to fully verify a system in this way.

A \emph{formal specfication} is a specification written in a \emph{formally
specified language}. The formal nature of the specification allows one to
reason about system correctness using formal methods, e.g. mathematical
analysis.

A correct system is not necessarily efficient, but an efficient system is
desired in practice. A correct system may therefore be \emph{refined} to a more
efficient system, or an already efficient system may be \emph{verified} to be
equivalent to an already correct system. In the latter case, the already
efficient system itself needs to be formally specified for the verification to
make sense.

