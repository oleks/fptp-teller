\documentclass[a4paper]{article}

\usepackage[fancy]{template}
\usepackage{survival-pack}

\setup{%
  subject={Master Project},%
  assignment={Project Description},%
  date={October 19, 2013.}%
}
\setupLocation[short=DIKU]{Datalogisk institut, Copenhagen University}
\setupAuthor[addendum={\email{oleks@diku.dk}}]{Oleksandr Shturmov}

\begin{document}

\maketitle
\thispagestyle{first} % no fancy header on first page (just a fancy footer).

A first past the post (FPTP) election is won by the candidate(s) obtaining a
majority of the votes. The majority may be relative (most votes) or absolute
($> 50\%$). In case of ties, or lack of winners, runoff elections may be held
until a single winner is attained. A voter may vote for exactly one candidate,
typically in an anonymous manner.

FPTP is most common in states that descended from the British and French
empires, e.g. UK, United States, Canada, or India. It is the second most widely
used election system in the world, after party-list proportional representation
(PLPR). FPTP is often used for head of state elections, while PLPR is often
used for parliamentary elections. FPTP can also be used for parliamentary
elections having single-member elective districts.

Once the votes have been cast, a mission critical process is necessary to count
the votes. This is done by a tally system. The purpose of this project is to
design an open source, automated tally system for FPTP elections, formally
specified in Z, implemented and verified in SPARK. This is further explained
below.

\section*{Formal methods}

% new concepts, or concepts that the reader may understand differently are
% emphasized.

A \emph{system} is a set of data structures and transformations for solving a
particular set of problems. A system \emph{specification} is a document
describing the system. A \emph{specification language} is the language used to
write a specification.

A \emph{formal specfication} is a specification written in a \emph{formally
specified language}. The formal nature of the specification allows one to
reason about the system using formal methods, e.g. mathematical analysis.

Every program is a specification, and every programming language is a
specification language. A specification is \emph{efficient} if it is directly
executable by a computer. A specification need not be efficient in general.

For instance, the specification language may allow for the use of certain
mathematical constructs. This would allow to disregard certain data
representation and transformation details. These details are important for an
efficient implementation, but may get in the way of reasoning about system
\emph{correctness}.

We reason about system correctness, by stating certain \emph{properties} that
should hold, and \emph{verifying} that they indeed do.

One way to do this is to verify system behaviour. That is, for all possible
input and generate the expected output, and check that the system produces the
same result. As the problem space explodes (as it quickly does), it becomes
infeasible to fully verify a system in this way.

With a formal specification, we can reason about the specification rather than
the system behaviour. For instance, if the specification language allows the
use of certain mathematical constructs, certain mathematical techniques can be
used to verify that the system satisfies the formally stated properties.

A correct system is not necessarily efficient, but an efficient system is
desired in practice. A correct system may therefore be \emph{refined} to a more
efficient system, or an already efficient system may be \emph{verified} to be
equivalent to an already correct system. In the latter case, the already
efficient system itself needs to be formally specified for the verification to
make sense.



\section*{Z}

% new concepts, or concepts that the reader may understand differently are
% emphasized.

Z is a formally specified language for writing formal specifications. Its
syntax, type system and semantics are formally defined by \cite{iso-z}. A
tutorial and a de-facto specification is provided by M. Spivey in \cite{zrm}.

Z builds upon the Zermelo-Frankel set theory with the axiom of choice (ZFC), a
one-sorted theory of first-order logic. ZFC is an axiomatic set theory, i.e. a
theory of sets which avoids the paradoxes of na\"ive set theory due to
unrestricted comprehension.



\section*{SPARK}

SPARK is a formally-specified programming language for writing formally
verified programs\cite{spark-lrm}. A comprehensive tutorial is provided by John
Barnes in \cite{barnes}.

SPARK is (1) a conservative subset of the Ada programming language, together
with (2) annotations as Ada comments, and (3) tools for verification of SPARK
programs. The annotations provide the semantic information necessary to guide
the verification. Verification is the process of confirming that the Ada code
enclosed in an annotation conforms to the annotation. This in turn may require
intermediate annotations, e.g. loop invariants.

The process of verification is only partially automated, but it can be checked
automatically whether a SPARK system is fully verified or not.

% a system with a low level of abstraction makes analysis difficult, as
% properties cannot be expressed compactly, if at all.


% Copyright (c) 2013 Oleksandr Shturmov.
% Licenced under the EUPL, Version 1.1 only.
% You may obtain a copy of the Licence at:
% http://joinup.ec.europa.eu/software/page/eupl/licence-eupl

\begin{thebibliography}{9} % 9 if < 10 reference, 99 if < 100 references, etc.

\bibitem[Z Reference Manual]{zrm}

J. M. Spivey. Programming Research Group. University of Oxford. The Z Notation:
A Reference Manual. 1998, 2nd ed. Published by J.M. Spivey, Oriel College,
Oxford,OX1 4EW, England.  Retrieved from
\url{http://spivey.oriel.ox.ac.uk/~mike/zrm/zrm.pdf} on May 11, 2013.

\end{thebibliography}



\end{document}

