% vim: set spell:

% Copyright (c) 2013 Oleksandr Shturmov.
% Licenced under the EUPL, Version 1.1 only.
% You may obtain a copy of the Licence at:
% http://joinup.ec.europa.eu/software/page/eupl/licence-eupl

\section{Social Choice}

\def\pref{\succsim}
\def\npref{\nsuccsim}
\def\spref{\succ}
\def\nspref{\nsucc}
\def\indif{\sim}
\def\nindif{\nsim}

Let a \emph{world} be a set of elements in a particular \emph{state}. Let an
\emph{alternative} be a collective state of a subset of the world. Let an
\emph{individual} be a subset of the world that prefers some alternatives over
others.

\emph{Social choice} is the study of deducing collective preferences from
individual preferences.

A preference may be \emph{oridnal} or \emph{cardinal}, in the sense that a
certain alternative may simply be preferred to another, or be preferred by a
certain amount. The latter requires a preference metric upon which all
individuals agree. This itself is a matter of social choice, and is susceptible
to philosophical debate, so we will not be concerned with it here.

Let preferences be ordinal, and thereby amenable to representation using binary
relations. Let $A$ denote a set of alternatives. Let $\pref$ (pronounced ``is
preferred to'') be an infix operator denoting a preference relation. Assuming a
set of tuples representation of relations, we have $\pref \in
\mathbb{P}\p{A\times A}$.  We'll assume that preference relations are
\emph{transitive} and \emph{total}, i.e.  $\forall a,b,c\in A \bullet a\pref
b \wedge b \pref c \Rightarrow a \pref c$ and $\forall a,b\in A \bullet a \pref
b \vee b \pref a$.

We'll introduce two admissible relations for any given preference relation:
\emph{strict preference} and \emph{indifference}. Let $\spref$ (pronounced ``is
strictly preferred to'') be the relation $\set{\p{a,b} : A \times A \st{a \pref
b \wedge b \npref a}}$. Let $\indif$ (pronounced ``is indifferent to'') be the
relation $\set{\p{a,b} : A \times A \st{a \pref b \wedge b \pref a}}$.

Let $N=\set{i:\mathbb{N}\st{i<n}}$ be the set of $n\in\mathbb{N}$ individuals
in a collective.  Let $P=\set{\pref_i\st{i\in \mathbb{N} \wedge i < n}}$ be the
set of their individual preference relations, and $\pref$ be their collective
preference relation.  Likewise, for the relations $\spref$ and $\indif$.

% The way that we show that these properties do not hold is by showing a toy
% election where they do not hold.

% In all scoring rules, the last alternative is ranked 0.
%
% Count the number of voters that picked some alternative first (times 1,
% normalize it all by |U|-1.
% 
% Count the number of voters that picked some alternative second (variable
% s_2). s_2 (after normalization) is in the range 0 to 1.
% 

\subsection{Further Reading}

This section does not provide a thorough introduction to social choice. This
section merely introduces those concepts of social choice relevant for our
further discussions. The interested reader is welcomed to further readings on
the subject.

For a semi-formal, philosophical discussion on the matter of preference, and
ultimately social choice, consider the article \cite{preferences} in the
Stanford Encyclopedia of Philosophy.

For an introduction from a computer science perspective, the reader is referred
to the relatively young field of Computational Social Choice, e.g.
\cite{brandt, chevaleyre}.

For a comprehensive treatment of social choice theory, with a focus on social
choice functions, simple majorities, the Condorcet criterion, and others, the
reader is referred to Peter C. Fishburn's The Theory of Social Choice
\cite{fishburn}.

