% vim: set spell:

% Copyright (c) 2013 Oleksandr Shturmov.
% Licenced under the EUPL, Version 1.1 only.
% You may obtain a copy of the Licence at:
% http://joinup.ec.europa.eu/software/page/eupl/licence-eupl

\section{Social Choice}

\def\pref{\succsim}
\def\npref{\nsuccsim}
\def\spref{\succ}
\def\nspref{\nsucc}
\def\indif{\sim}
\def\nindif{\nsim}

Let a \emph{world} be a set of elements in a particular \emph{state}. Let an
\emph{alternative} be a collective state of a subset of the world. Let an
\emph{individual} be a subset of the world that prefers some alternatives over
others.

\emph{Social choice} is the study of transformation of individual preferences
into collective preferences. It is important to note that a collective in and
of itself does not possess preferences. Social choice, therefore, also deals
with the philosophical problem inherent in deducing collective preferences.

A preference may be \emph{oridnal} or \emph{cardinal}, in the sense that a
certain alternative may simply be preferred to another, or be preferred by a
certain amount. The latter requires a preference metric upon which all
individuals agree. This itself is a matter of social choice, and is susceptible
to philosophical debate, so we will not be concerned with it here.

Let preferences be ordinal, and thereby amenable to representation using binary
relations. Let $A$ denote a non-empty set of alternatives. Let $\pref$
(pronounced ``is preferred to'') be an infix operator denoting a preference
relation. Assuming a set of tuples representation of relations, we have $\pref
\in \mathbb{P}\p{A\times A}$.  We'll assume that preference relations are
\emph{transitive} and \emph{complete}, i.e.  $\forall a,b,c\in A \bullet a\pref
b \wedge b \pref c \Rightarrow a \pref c$ and $\forall a,b\in A \bullet a \pref
b \vee b \pref a$.

We'll introduce two admissible relations for any given preference relation:
\emph{strict preference} and \emph{indifference}. Let $\spref$ (pronounced ``is
strictly preferred to'') be the relation $\set{\p{a,b} : A \times A \st{a \pref
b \wedge b \npref a}}$. Let $\indif$ (pronounced ``is indifferent to'') be the
relation $\set{\p{a,b} : A \times A \st{a \pref b \wedge b \pref a}}$.

\subsection{Further Reading}

This section does not provide a thorough introduction to Social Choice. This
section merely introduces those concepts of Social Choice relevant for our
further discussion. The interested reader is therefore guided to further
readings on the subject.

For a formal, philosophical discussion on the matter of preference, and
ultimately social choice, consider the article \cite{preferences} in the
Stanford Encyclopedia of Philosophy.



