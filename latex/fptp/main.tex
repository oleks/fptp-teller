% vim: set spell:

% Copyright (c) 2013 Oleksandr Shturmov.
% Licenced under the EUPL, Version 1.1 only.
% You may obtain a copy of the Licence at:
% http://joinup.ec.europa.eu/software/page/eupl/licence-eupl

\chapter{FPTP}

A First-Past-The-Post (FPTP) election is among the simplest types of elections.
It is easy to understand, conduct and monitor on a small scale.

In its simplest incarnation, it is the ``raising of hands'' election. The
winner is the candidate with the highest number of raised hands. To ensure
anonimity, the voter may instead be asked to denote his choice on a piece of
paper, and cast it in a ballot box. Once all the votes have been cast, the
votes are tallied, and the candidate most frequently denoted, wins the
election.

The name ``first-past-the-post'' stems from horse racing, where the first horse
past a particular post wins the race. Unlike in horse racing however, there is
often a high chance of a tie in an FPTP election.

On a national scale, an FPTP ballot is typically a list of candidates written
out on a sheet of paper, and the voter is asked to mark their desired
candidate, and cast the ballot into a ballot box. The bllots in all the ballot
boxes are then tallied, and the candidate with the most marks wins the
election.

% vim: set spell:

\section{Social Choice}

Let a \emph{world} be a set of elements in a particular \emph{state}. Let an
\emph{alternative} be a collective state of a subset of the world. Let an
\emph{individual} be a subset of the world that possesses preference relations
over alternatives, i.e. prefers some alternatives to others. Let a
\emph{collective} be a set of individuals. A collective may, likewise, possess
preference relations over alternatives.

\emph{Social choice} is the study of transformation of individual preference
relations into collective preference relations.



