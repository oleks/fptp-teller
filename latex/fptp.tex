\chapter{FPTP}

A First-Past-The-Post (FPTP) election is among the simplest types of elections.
It is easy to understand, conduct and monitor on a small scale.

This is the common man's raising of hands election. The winner is the candidate
with the highest number of raised hands. In an anonymous FPTP election, a voter
may be asked to write their desired candidate on a sheet of paper, and cast it
into a container. Once all the votes have been cast, the votes are tallied, and
the candidate with the most sheets of paper with her name on it, wins the
election.

The name ``first-past-the-post'' stems from horse racing, where the first horse
past a particular post wins the race. Unlike in horse racing however, there is
often a high chance of a tie in an FPTP election.

On a national scale, an FPTP ballot is typically a list of candidates written
out on a sheet of paper, and the voter is asked to mark their desired
candidate, and cast the ballot into a ballot box. The ballots in all the ballot
boxes are then tallied, and the candidate with the most marks wins the
election.

In an FPTP election, the voter is handed a ballot with a list of candidates, and is typically asked to mark

It is commonly
employed at various small council elections, in the common man's raising of
hands form.

commonly employed in
household elections.
