\chapter{Z}

Votes are cast and tallied anonymously using a token system. A voter may chose
among a set of available of options to express her intent. We assume a set of
voter and option tokens as basic types. Issuing tokens is beyond the scope of
this specification.

\begin{zed}
[VOTER, OPTION]
\end{zed}

In an election, individuals are registered as voters, and candidates are
registered as options. The process of voter and candidate registration is
beyond the scope of this specification. We assume that everyone eligible to
vote is registered as a voter, and everyone eligible to be elected is
registered as a candidate. We assume a nonempty finite set of registered voters
and candidates:

\begin{axdef}
$voters : \finset_1 VOTER$ \\
$candidates: \finset_1 OPTION$
\end{axdef}

Not all voter and option tokens are necessarily dealt. Token generation can
happen offline, prior to registration, or online, during registration. It is
beyond the scope of this specification to ensure that enough tokens are
generated to accommodate all the registrations.

In an FPTP election, a voter always chooses one among her available options.
The options are first and foremost a list of candidates. This list may be
insufficient to express voter intent, and a ``none of the above'' option is
typically added.

Aside from casting a valid vote, the voter may also:

\begin{enumerate}

\item not cast a vote;

\item cast a blank vote; or

\item cast an invalid vote.

\end{enumerate}

Depending on the legislature, the above options may have different effects on
the tally. For instance, not casting a vote may be legislatively equivalent to
voting for ``none of the above''. We leave it to further postprocessing to
handle such discrepancies. We regard all of the above as separate options, and
present a tally for each option.

\begin{axdef}
$nota, absent, blank, invalid : OPTION$ \\
$options : \finset_1 OPTION $
\where
$nota, novote, blank, invalid \notin candidates $\\
$options = \set{ nota, novote, blank, invalid } \cup candidates$
\end{axdef}

Only registered voters can cast votes. A vote is a particular option chosen by
a particular voter. We choose to encode the ballots in the system as a
nonpartial function with adequately restricted types:

\begin{axdef}
$ballots : voters \fun options$
\end{axdef}

Note, a voter always has a chosen option, even before an election occurs. In
particular, before any votes are cast, all voters are $absent$. TODO: ballot
registration.

After the voters, candidates, and ballots are in, we turn to the tallying
itself.

The teller computes a tally for every option. Some of those options are
candidates. The winners are those candidates that have attained a tally greater
than or equal to every other candidate.

\begin{axdef}
  $tally : options \fun \nat$ \\
  $winners : \finset_1 candidates$
\where
  $\forall\ o : options \bullet tally\p{o} =
    \# \p{ ballots \vartriangleleft \set{ o } }$ \\
  $\forall\ w : winners \bullet \forall\ c : candidates
   \bullet tally\p{w} \geq tally\p{c}$ \\
\end{axdef}

TODO: $winners$ is a bad name. They're more like relative front runners.

Corollaries:

\begin{axdef}
$\# \p{ winners } \leq \# \p{ candidates }$ \\
\end{axdef}

\begin{axdef}
$\# \p{ ballots } = \# \p{ voters }$
\end{axdef}

To show:

\begin{axdef}
$\# \p{ candidates } = 1 \Rightarrow \# \p{ winners } = 1$
\end{axdef}

