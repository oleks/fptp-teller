\chapter{Z}

Votes are cast and tallied anonymously using a token system. Issuing tokens is
beyond the scope of this specification. A voter may chose among a set of
available of options to express her intent. We assume a set of voter and option
tokens as basic types:

\begin{zed}
[VOTER, OPTION]
\end{zed}

In an election, individuals are registered as voters and candidates are
registered as options. The process of voter and candidate registration is
beyond the scope of this specification. We assume that everyone eligible to
vote is registered as a voter. We assume a nonempty finite set of registered
voters and candidates:

\begin{axdef}
$voters : \finset_1 VOTER$ \\
$candidates: \finset_1 OPTION$
\end{axdef}

Not all voter and option tokens are necessarily dealt. Token generation can
happen offline, prior to registration, or online, during registration. It is
beyond the scope of this specification to ensure that enough tokens are
generated to accommodate all the registrations.

In an FPTP election, a voter chooses one among the available options on a
ballot. An FPTP ballot is first and foremost a list of candidates. This list
may be insufficient to express voter intent. For this purpose, a ``none of the
above'' option is typically added to the ballot.

Aside from casting a valid vote, the voter may also:

\begin{enumerate}

\item not cast a vote;

\item cast a blank vote; or

\item cast an invalid vote.

\end{enumerate}

We take these options into consideration and say that a voter always chooses
one among her possible options, in particular, a voter may chose the option to
not cast a vote. A voter cannot not choose one of the options.

Depending on the legislature, some, or all of the above may be equivalent to
voting for ``none of the above'', or casting an invalid vote. Likewise, it
depends on the legislature what effect such votes (or lack thereof) have on the
tally. To provide for these options, we introduce the following global
variables:

%\begin{axdef}
%hasNota : \set{0, 1}
%\end{axdef}
