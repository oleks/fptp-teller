% Copyright (c) 2013 Oleksandr Shturmov.
% Licenced under the EUPL, Version 1.1 only.
% You may obtain a copy of the Licence at:
% http://joinup.ec.europa.eu/software/page/eupl/licence-eupl

\begin{thebibliography}{9} % 9 if < 10 references, 99 if < 100 references, etc.

\bibitem[Preferences]{preferences}

Hansson, Sven Ove and Grüne-Yanoff, Till. \emph{Preferences}. The Stanford
Encyclopedia of Philosophy (Winter 2012 Edition), Edward N. Zalta (ed.),
\url{http://plato.stanford.edu/archives/win2012/entries/preferences/}.

\bibitem[Brandt, et al.]{brandt}

F. Brandt, V. Conitzer, and U. Endriss. \emph{Computational Social Choice}. To
appear in G. Weiss (Ed.), Multiagent Systems, MIT Press, 2012.

\bibitem[Chevaleyre, et al.]{chevaleyre}

Yann Chevaleyre, Ulle Endriss, Jérôme Lang, Nicolas A Maudet. \emph{A Short
Introduction to Computational Social Choice}. SOFSEM 2007: Theory and Practice
of Computer Science. Lecture Notes in Computer Science. Springer Berlin
Heidelberg. ISBN 978-3-540-69506-6. pp. 51--69.

\bibitem[Fishburn]{fishburn}

Peter C. Fishburn. \emph{The Theory of Social Choice}. 1973. Princeton
University Press. ISBN 978-0-691-08121-2.

\bibitem[IDEA Handbook]{idea-handbook}

Andrew Reynolds, Ben Reilly, Andrew Ellis, et al. Institute for Democracy and
Electoral Assitance. \emph{Electoral System Design: The New International IDEA
Handbook}. 2008. Trydells Tryckeri AB, Sweden. ISBN: 91-85391-18-2. Retrieved
from \url{http://www.idea.int/publications/esd/} on August 7, 2013.

\emph{A handbook covering the essentials of electoral systems, their benefits,
and pitfalls.}

\bibitem[Z Reference Manual]{zrm}

J. M. Spivey. Programming Research Group. University of Oxford. \emph{The Z
Notation: A Reference Manual}. 1998, 2nd ed. Published by J.M. Spivey, Oriel
College, Oxford, OX1 4EW, England. Retrieved from
\url{http://spivey.oriel.ox.ac.uk/~mike/zrm/} on May 11, 2013.

\emph{The canonical Z reference manual. Seconded only by the Z ISO standard
itself.}

\end{thebibliography}

